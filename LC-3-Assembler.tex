\documentclass{article}
\usepackage[utf8]{inputenc}
\usepackage{tikz}
\usepackage{indentfirst}
\usepackage[letterpaper, portrait, margin=1in]{geometry}
\usepackage{array}
\usepackage{amsmath}
\usepackage{soul}
\usepackage{todonotes}
\usepackage{textcomp}


\setlength{\parindent}{0pt}

\title{CS 2110 LC-3 Assembler Example}
\author{Pulkit Gupta}
\date{October 16, 2019}

\begin{document}

\maketitle

\tableofcontents
\section{What is this assembler}
I recommend reading the lab guide for lab 14 to familiarize yourself with how the LC-3 assembler works.
\\\\
This assembler (LC3asm.java) is a simple 2 pass assembler that performs most of the features of the actual LC-3 assembler as described in chapter 7 of the textbook. The whole purpose of releasing this assembler to students is to give you the ability to read the source of a 2 pass assembler and see how it would work.
\\\\
Key features that are missing/changed include:
\begin{itemize}
    \item aliases for traps are not implemented (IN, OUT, PUTS, etc.)
    \item immediate values and PC relative offsets are not checked for validity as to make the code a little easier to read
    \item the obj file output does not list the number of instructions at every .ORIG statement, instead it says "ORIG: x\#\#\#\#" to indicate where a block of memory goes.
\end{itemize}

\subsection{What it does}
Given a valid input file, this assembler will generate 3 files.
\begin{itemize}
    \item .debug - holds debugging and assembly process information (similar to .lst)
    \item .sym - the symbol table generated by the first pass
    \item .obj - stores the .orig addresses and hex of each instruction/memory address relevant to the program, intended for you to be able to use in a simulator
    \item .dat - stores the program as a format ready for use in the datapath assignment in case you want to write your own test cases
\end{itemize}

\subsection{Using this Assembler}
You can compile the assembler with: "javac LC3asm.java"
\\\\
You can run it on a file with: "java LC3asm \textless my-file.asm\textgreater"
\\\\
Once it finishes, you should be able to check out the files it generated, my-file.debug, my-file.sym, my-file.obj, my-file.dat

\end{document}